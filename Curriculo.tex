\documentclass[a4paper,12pt]{article}
\usepackage[brazil]{babel}
\usepackage[utf8]{inputenc}
\usepackage[T1]{fontenc}
\usepackage{geometry}
\usepackage{enumitem}
\usepackage{hyperref}
\usepackage{xcolor}
\usepackage{titlesec}
\usepackage{fontawesome}
\geometry{left=2cm, right=2cm, top=2cm, bottom=2cm}
\titleformat{\section}{\large\bfseries}{}{0em}{}[\titlerule]

\begin{document}

\begin{center}
    {\LARGE \textbf{Manuel de Vooght}} \\
    Rio de Janeiro, Brasil \\
    \href{mailto:m.guilhermevooght@gmail.com}{m.guilhermevooght@gmail.com} \\
    \href{https://www.linkedin.com/in/mvooght}{\color{blue}\faLinkedin \space LinkedIn}
\end{center}

\vspace{1em}

\section*{Resumo Profissional}
Atuo como consultor estatístico, fornecendo suporte especializado em análise de dados, modelagem estatística avançada e desenvolvimento de relatórios para embasar a tomada de decisão em projetos científicos e empresariais. Tenho experiência com modelos estatísticos complexos, incluindo modelos mistos lineares, log-binomial e multinomial, bem como análise epidemiológica e populacional. Minha atuação também inclui cálculos de taxas ajustadas, séries temporais (Joint Point Regression) e análises espaciais (regressão e autocorrelação).

Além disso, estruturo e organizo bancos de dados, automatizo processos analíticos para otimizar eficiência e desenvolvo dashboards estratégicos para visualização de dados. Minha experiência em gestão de projetos garante a entrega de análises precisas e suporte em todas as etapas do fluxo de trabalho, desde a coleta e organização dos dados até a geração de insights e apresentação de resultados. Trabalho com R, Python, Excel, QGIS, Power BI e SaTScan.

\vspace{0.6em}

\section*{Formação Acadêmica}
\textbf{Universidade do Estado do Rio de Janeiro (UERJ)} \\
\textit{Mestrado em Saúde Coletiva} – IMS/UERJ (2022) \\
\textit{Graduação em Ciências Sociais} – ICS/UERJ (2019)

\vspace{0.6em}

\section*{Experiência Profissional}
\textbf{BiomedStat} – Desde 2024 \\
\textit{Consultor em Análises Estatísticas} \\
\begin{itemize}[leftmargin=*]
    \item Consultoria em análise estatística e interpretação de dados para projetos na área de saúde e biomedicina.
    \item Desenvolvimento de modelos estatísticos e técnicas avançadas para análise de dados epidemiológicos e populacionais.
    \item Estruturação, organização e automação de bancos de dados para otimização de processos analíticos.
    \item Criação de dashboards interativos para visualização estratégica de informações.
\end{itemize}

\textbf{Instituto Brasileiro de Geografia e Estatística (IBGE)} – Julho 2022 a Junho 2023 \\
\textit{Recenseador – Censo 2022} \\
\begin{itemize}[leftmargin=*]
    \item Coleta e análise de dados estatísticos para pesquisas socioeconômicas e demográficas.
    \item Garantia da qualidade e precisão na inserção de informações e tratamento de inconsistências.
\end{itemize}

\textbf{Universidade do Estado do Rio de Janeiro (UERJ)} – Centro Brasil de Saúde Global – IMS/UERJ \\
\textit{Estagiário de Extensão} – Junho 2017 a Março 2019 \\
\begin{itemize}[leftmargin=*]
    \item Organização da disciplina "Panoramas da Saúde Global" e suporte técnico.
    \item Gravação e edição de aulas, manutenção de website e atualização de mídias sociais.
\end{itemize}

\textbf{Universidade do Estado do Rio de Janeiro (UERJ)} – Departamento de Cooperação Internacional \\
\textit{Estagiário} – Abril 2012 a Agosto 2013 \\
\begin{itemize}[leftmargin=*]
    \item Contato com universidades estrangeiras e apoio a alunos de intercâmbio.
    \item Elaboração de documentos administrativos e recepção de delegações internacionais.
\end{itemize}

\vspace{0.6em}

\section*{Intercâmbio}
\textbf{Universidad de Jaén (Andaluzia, Espanha)} – Agosto 2013 a Agosto 2014 \\
Centro de Estudios Avanzados en Lenguas Modernas.

\vspace{0.6em}

\section*{Habilidades e Ferramentas}
\begin{itemize}[leftmargin=*]
    \item Modelagem estatística avançada e análise de dados epidemiológicos.
    \item Programação e automação de processos analíticos em R e Python.
    \item Organização e estruturação de bancos de dados complexos.
    \item Desenvolvimento de dashboards interativos no Power BI para visualização de insights estratégicos.
    \item Análises espaciais e epidemiológicas utilizando QGIS e SaTScan.
\end{itemize}

\vspace{0.6em}

\section*{Idiomas}
\begin{itemize}[leftmargin=*]
    \item Inglês Avançado (C1) – Cultura Inglesa (2004-2011, concluído).
    \item Espanhol Avançado (C1) – Universidad de Jaén (2013-2014, concluído).
\end{itemize}

\end{document}
